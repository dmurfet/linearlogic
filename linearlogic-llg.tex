\documentclass[english,letter paper,12pt,reqno]{article}
\usepackage{etex} % to fix "no room for new dimen" error. Voodoo.
\usepackage{array}
\usepackage{bussproofs}
\usepackage{epigraph}
\usepackage{stmaryrd}
\usepackage{mathdots}
\usepackage{amsmath, amscd, amssymb, mathrsfs, accents, amsfonts,amsthm}
\usepackage[all]{xy}
\usepackage{mathtools} % for bra-ket
\usepackage{tikz}
\usetikzlibrary{calc}

\AtEndDocument{\bigskip{\footnotesize%
  \textsc{Department of Mathematics, University of Southern California} \par  
  \textit{E-mail address}: \texttt{murfet@usc.edu} \par
}}

% Labels in tabular
\newcommand{\tagarray}{\mbox{}\refstepcounter{equation}$(\theequation)$}

% Bra-ket stuff
\DeclarePairedDelimiter\bra{\langle}{\rvert}
\DeclarePairedDelimiter\ket{\lvert}{\rangle}
\DeclarePairedDelimiterX\braket[2]{\langle}{\rangle}{#1 \delimsize\vert #2}
\DeclarePairedDelimiterX\inner[2]{\langle}{\rangle}{#1,#2}
\DeclarePairedDelimiter\abs{\lvert}{\rvert}
\DeclarePairedDelimiter\norm{\lVert}{\rVert}
\DeclarePairedDelimiter\set{\lbrace}{\rbrace}

%\begin{align*}
%\bra{a}       &= \bra*{\frac{a}{1}}\    \ket{a}       &= \ket*{\frac{a}{1}}\    \braket{a}{b} &= \braket*{\frac{a}{1}}{\frac{b}{1}}\    %\inner{a}{b}  &= \inner*{\frac{a}{1}}{\frac{b}{1}}\    \abs{a}       &= \abs*{\frac{a}{1}}\    \norm{a}      &= \norm*{\frac{a}%{1}}\    \set{a,b}     &= \set*{\frac{a}{1},\frac{b}{1}}
%\end{align*}

% TikZ stuff
\def\drawbang{\draw[color=teal!50, line width=2pt]}
\def\drawprom{\draw[color=gray, line width=3pt]}
\def\bluenode{\node[circle,draw=blue!50,fill=blue!20]}
\def\mapnode{\node[circle,draw=black,fill=black,inner sep=0.5mm]}
\def\whitenode{\node[circle,draw=blue!50,fill=blue!5]}
\def\dernode{\node[circle,draw=black,fill=white]}
\definecolor{Myblue}{rgb}{0,0,0.6}
\usepackage[a4paper,colorlinks,citecolor=Myblue,linkcolor=Myblue,urlcolor=Myblue,pdfpagemode=None]{hyperref}

\SelectTips{cm}{}

\setlength{\evensidemargin}{0.1in}
\setlength{\oddsidemargin}{0.1in}
\setlength{\textwidth}{6.3in}
\setlength{\topmargin}{0.0in}
\setlength{\textheight}{8.5in}
\setlength{\headheight}{0in}

\newtheorem{theorem}{Theorem}[section]
\newtheorem{proposition}[theorem]{Proposition}
\newtheorem{lemma}[theorem]{Lemma}
\newtheorem{corollary}[theorem]{Corollary}

\newtheoremstyle{example}{\topsep}{\topsep}
	{}
	{}
	{\bfseries}
	{.}
	{2pt}
	{\thmname{#1}\thmnumber{ #2}\thmnote{ #3}}
	
	\theoremstyle{example}
	\newtheorem{definition}[theorem]{Definition}
	\newtheorem{example}[theorem]{Example}
	\newtheorem{remark}[theorem]{Remark}
	\newtheorem{question}[theorem]{Question}

\numberwithin{equation}{section}

% Operators
\def\eval{\operatorname{ev}}
\def\sh{\operatorname{Sh}}
\def\res{\operatorname{Res}}
\def\Coker{\operatorname{Coker}}
\def\Ker{\operatorname{Ker}}
\def\im{\operatorname{Im}}
\def\can{\operatorname{can}}
\def\K{\mathbf{K}}
\def\D{\mathbf{D}}
\def\N{\mathbf{N}}
\def\LG{\mathcal{LG}}
\def\Ab{\operatorname{Ab}}
\def\Hom{\operatorname{Hom}}
\def\modd{\operatorname{mod}}
\def\Modd{\operatorname{Mod}}
\DeclareMathOperator{\Ext}{Ext}
\DeclareMathOperator{\Tr}{Tr}
\DeclareMathOperator{\End}{End}
\DeclareMathOperator{\rank}{rank}
\DeclareMathOperator{\tot}{Tot}
\DeclareMathOperator{\ch}{ch}
\DeclareMathOperator{\str}{str}
\DeclareMathOperator{\hmf}{hmf}
\DeclareMathOperator{\HMF}{HMF}
\DeclareMathOperator{\hf}{HF}
\DeclareMathOperator{\At}{At}
\DeclareMathOperator{\Cat}{Cat}
\DeclareMathOperator{\Spec}{Spec}
\DeclareMathOperator{\MSpec}{MSpec}
\DeclareMathOperator{\Sym}{Sym}
\DeclareMathOperator{\LC}{LC}
\def\inta{\bold{int}}
\def\comp{\underline{\textup{comp}}}
\def\contract{\;\lrcorner\;}

% SCprooftree environment
\newenvironment{scprooftree}[1]
{\gdef\scalefactor{#1}\begin{center}\proofSkipAmount \leavevmode}
{\scalebox{\scalefactor}{\DisplayProof}\proofSkipAmount \end{center} }
  
\begin{document}

% Bussproof things
\def\ScoreOverhang{1pt}

% Commands
\def\Res{\res\!}
\newcommand{\ud}[1]{\operatorname{d}\!{#1}}
\newcommand{\Ress}[1]{\res_{#1}\!}
\newcommand{\cat}[1]{\mathcal{#1}}
\newcommand{\lto}{\longrightarrow}
\newcommand{\xlto}[1]{\stackrel{#1}\lto}
\newcommand{\mf}[1]{\mathfrak{#1}}
\newcommand{\md}[1]{\mathscr{#1}}
\newcommand{\church}[1]{\underline{#1}}
\newcommand{\prf}[1]{\underline{#1}}
\newcommand{\den}[1]{\llbracket #1 \rrbracket}
\def\l{\,|\,}
\def\sgn{\textup{sgn}}
\def\cont{\operatorname{cont}}

\title{Landau-Ginzburg semantics of linear logic}
\author{Daniel Murfet}

\maketitle

\section{Introduction}

In this paper we show that there is a natural connection between moduli spaces and logic, using the example of Landau-Ginzburg models. The bicategory of Landau-Ginzburg models $\LG$ has as objects isolated hypersurface singularities, and as $1$-morphisms matrix factorisations of differences of potentials. In \cite{??} this bicategory was studied in connection with topological field theory with defects and shown to have various good properties, and in \cite{??} it was shown how to make composition in this bicategory constructive. In this paper we extend this work by adding \emph{moduli spaces} of matrix factorisations, and showing how the resulting bicategory hosts a model (called a semantics) of linear logic.

\begin{itemize}
\item Landau-Ginzburg models
\item Algorithmically computable
\item Example of a self-composition of rank $1$ endo-MFs. This includes $P_S$'s so is already plenty complicated.
\item Linear logic
\item Semantics of linear logic
\item Moduli spaces, simple example
\item Bundle over moduli space of self-compositions
\item This is the denotation of the proof $2$. 
\item Things get complicated: $\inta_{\inta_x}$
\end{itemize}



%The abstract syntax of a computer program, written in a high level language like Java or Haskell, is executed on a physical machine after being translated to commands for the operation of the hardware (e.g. moving a memory word between registers). This \emph{operational semantics} is too ad-hoc and tied to a particular machine to allow for the easy recognition of the mathematical properties of programs, and this is the impetus for the study of intermediate \emph{denotational semantics} where the structure of a program is faithfully reflected in some other mathematical structure. 

% When the program is described in certain logics, such as linear logic, the mathematical structures may be particular kinds of symmetric monoidal categories. In this paper we describe a denotational semantics of linear logic where each \emph{type} $A$ is mapped to a tuple $\den{A} = (X, Y, W)$ where $X$ is an algebraic variety, $Y$ an affine space, and $W$ is a global section of $Y$, and a \emph{program} $\pi$ of type $A$ is mapped to a tuple $\den{\pi} = (H, I, \md{E})$ where $H$ is an algebraic variety of ``hidden variables'', $I$ is a closed subscheme of $X \times H \times Y$ and $\md{E}$ is a matrix factorisation of $W$ on $X \times H \times Y$.

\bibliographystyle{amsalpha}
\providecommand{\bysame}{\leavevmode\hbox to3em{\hrulefill}\thinspace}
\providecommand{\href}[2]{#2}
\begin{thebibliography}{BHLS03}

\bibitem{abramsky}
S.~Abramsky, \textsl{Computational interpretations of linear logic}, Theoretical Computer Science, 1993.

\bibitem{abramsky4}
S.~Abramsky, \textsl{Retracting some paths in process algebra}, In CONCUR 96, Springer Lecture Notes in Computer Science \textbf{1119}, 1--17, 1996.

\bibitem{abramsky2}
S.~Abramsky, \textsl{Geometry of {I}nteraction and linear combinatory algebras}, Mathematical Structures in Computer Science, \textbf{12}, 625--665, 2002.

\bibitem{abramsky3}
S.~Abramsky and R.~Jagadeesan, \textsl{New foundations for the {G}eometry of {I}nteraction}, Information and Computation \textbf{111} (1), 53--119, 1994.

\bibitem{anel}
M.~Anel, A.~Joyal, \textsl{Sweedler theory of (co)algebras and the bar-cobar constructions}, \href{http://arxiv.org/abs/1309.6952}{[arXiv:1309.6952]}

\bibitem{atiyah}
M.~Atiyah, \textsl{Topological quantum field theories}, Publications Math\'{e}matique de l'IH\'{E}S 68, 175--186, 1989.

\bibitem{baez}
J.~Baez and M.~Stay, \textsl{Physics, topology, logic and computation: a Rosetta stone}, in B. Coecke (ed.) New Structures for Physics, Lecture Notes in Physics 813, Springer, Berlin, 95--174, 2011

\bibitem{barr}
M.~Barr, \textsl{Coalgebras over a commutative ring}, Journal of Algebra 32, 600--610, 1974.

\bibitem{barr_auto}
\bysame, \textsl{$\star$-autonomous categories}, Number 752 in Lecture Notes in Mathematics. Springer-Verlag, 1979.

\bibitem{barr_acc}
\bysame, \textsl{Accessible categories and models of linear logic}, Journal of Pure and Applied Algebra, 69(3):219--232, 1990.

\bibitem{barr_autolin}
\bysame, {$?$-autonomous categories and linear logic}, Mathematical Structures in Computer Science, 1(2):159--178, 1991.

\bibitem{barr_chu}
\bysame, \textsl{The {C}hu construction: history of an idea}, Theory and Applications of Categories, Vol. 17, No. 1, 10--16, 2006.

\bibitem{benton}
N.~Benton, \textsl{A mixed linear and non-linear logic; proofs, terms and models}, in Proceedings of Computer Science Logic 94, vol. 933 of Lecture Notes in Computer Science, Verlag, 1995.

\bibitem{benton_etal}
N.~Benton, G.~Bierman, V.~de Paiva and M.~Hyland, \textsl{Term assignment for intuitionistic linear logic}, Technical report 262, Computer Laboratory, University of Cambridge, 1992.

\bibitem{block-leroux} 
R.~Block, P.~Leroux, \textsl{Generalized dual coalgebras of algebras, with applications to cofree coalgebras}, J. Pure Appl. Algebra 36, no. 1, 15--21, 1985.

\bibitem{blute}
R.~Blute, \textsl{Hopf algebras and linear logic}, Mathematical Structures in Computer Science, 6(2):189--217, 1996.

\bibitem{blute_scott}
R.~Blute and P.~Scott, \textsl{Linear {L}a\"{u}chli semantics}, Annals of Pure and Applied Logic, 77:101--142, 1996.

\bibitem{blue_book}
\bysame, \textsl{Category theory for linear logicians}, Linear Logic in Computer Science 316: 3--65, 2004.

\bibitem{blute_fock}
R.~Blute, P.~Panangaden, R.~Seely, \textsl{Fock space: a model of linear exponential types}, in: Proc. Ninth Conf. on Mathematical Foundations of Programming Semantics, Lecture Notes in Computer Science, Vol. 802, Springer, Berlin, 1--25, 1994.

\bibitem{bott}
R.~Bott and L.W.~Tu, \textsl{Differential forms in {A}lgebraic {T}opology}, Graduate Texts in Mathematics, \textbf{82}, Springer, 1982.

\bibitem{ct1007.2679}
A.~{C\u ald\u araru} and S.~Willerton, \textsl{The Mukai pairing, I: a categorical approach},
New York Journal of Mathematics \textbf{16}, 61--98, 2010
  \href{http://arxiv.org/abs/0707.2052}{[arXiv:0707.2052]}.
  
\bibitem{lgdual}
N.~Carqueville and D.~Murfet, \textsl{Adjunctions and defects in {L}andau-{G}inzburg models}, \href{http://arxiv.org/abs/1208.1481}{[arXiv:1208.1481]}.

\bibitem{cr0909.4381}
N.~Carqueville and I.~Runkel, \textsl{On the monoidal structure of matrix bi-factorisations}, J. Phys.
  A: Math. Theor. \textbf{43} 275--401, 2010
  \href{http://arxiv.org/abs/0909.4381}{[arXiv:0909.4381]}.

\bibitem{church}
A.~Church, \textsl{The {C}alculi of {L}ambda-conversion}, Princeton University Press, Princeton, N. J. 1941.

\bibitem{danos}
V.~Danos and J.-B.~Joinet, \textsl{Linear logic and elementary time}, Information and Computation 183, 123--127, 2003.

\bibitem{danos_regnier1}
V.~Danos and L.~Regnier, \textsl{Local and {A}synchronous beta-reduction (an analysis of {G}irard's execution formula)} in: Springer Lecture Notes in Computer Science \textbf{8}, 296--306, 1993.

\bibitem{danos_regnier2}
V.~Danos and L.~Regnier, \textsl{Proof-nets and the Hilbert space}, in (Girard \textsl{et. al.} 1995), 307--328, 1995.

\bibitem{denning}
P.~J.~Denning, \textsl{Ubiquity symposium ``What is computation?''}: opening statement, Ubiquity 2010. Available on the \href{http://ubiquity.acm.org/article.cfm?id=1870596}{Ubiquity website}.

\bibitem{dm1102.2957}
T.~Dyckerhoff and D.~Murfet, \textsl{Pushing forward matrix factorisations}, Duke Math. J. Volume 162, Number 7 1249--1311, 2013 \href{http://arxiv.org/abs/1102.2957}{[arXiv:1102.2957]}.

\bibitem{ehrhard}
T.~Ehrhard, \textsl{Finiteness spaces}, Math. Structures Comput. Sci. 15 (4) 615--646, 2005.

\bibitem{ehrhard_kothe}
\bysame, \textsl{On {K}\"othe sequence spaces and linear logic}, Mathematical Structures in Computer Science 12.05, 579--623, 2002.

\bibitem{ehrhard_difflambda}
T.~Ehrhard and L.~Regnier, \textsl{The differential lambda-calculus}, Theoretical Computer Science 309.1: 1--41, 2003.

\bibitem{ehrhard_difflambda2}
\bysame, \textsl{Differential interaction nets}, Theoretical Computer Science 364.2: 166--195, 2006.

\bibitem{gentzen}
G.~Gentzen, \textsl{The Collected Papers of Gerhard Gentzen}, (Ed. M. E. Szabo), Amsterdam, Netherlands: North-Holland, 1969.

\bibitem{getzler}
E.~Getzler, P.~Goerss, \emph{A model category structure for differential graded coalgebras}, preprint, 1999.
  
\bibitem{girard_llogic}
J.-Y.~Girard, \textsl{Linear Logic}, Theoretical Computer Science 50 (1), 1--102, 1987.

\bibitem{girard_normal}
\bysame, \textsl{Normal functors, power series and the $\lambda$-calculus} Annals of Pure and Applied
Logic, 37: 129--177, 1988.

\bibitem{girard_goi1}
\bysame, \textsl{Geometry of {I}nteraction I: {I}interpretation of {S}ystem {F}}, in Logic Colloquium '88, ed. R.~Ferro, et al. North-Holland, 221--260, 1988.

\bibitem{girard_goi2}
\bysame, \textsl{Geometry of {I}nteraction II: {D}eadlock-free {A}lgorithms}, COLOG-88, Springer Lecture Notes in Computer Science \textbf{417}, 76--93, 1988.

\bibitem{girard_goi3}
\bysame, \textsl{Geometry of {I}nteraction III: {A}ccommodating the {A}dditives}, in (Girard \textsl{et al}. 1995), pp.1--42.

\bibitem{girard_towards}
\bysame, \textsl{Towards a geometry of interaction}, In J.~W.~Gray and A.~Scedrov, editors, Categories in Computer Science and Logic, volume 92 of Contemporary Mathematics, 69--108, AMS, 1989.

\bibitem{girard_complexity}
\bysame, \textsl{Light linear logic}, Information and Computation 14, 1995.

\bibitem{girard_coherentbanach}
\bysame, \textsl{Coherent {B}anach spaces: a continuous denotational semantics}, Theoretical Computer Science, 227: 275--297, 1999.

\bibitem{girard_blindspot}
\bysame, \textsl{The Blind Spot: lectures on logic}, European Mathematical Society, 2011.

\bibitem{girard_prooftypes}
J.-Y.~Girard, Y.~Lafont, and P.~Taylor, \textsl{Proofs and Types}, Cambridge Tracts in Theoretical Computer Science 7 ,Cambridge University Press, 1989.

\bibitem{Gonthier}
G.~Gontheir, M.~Abadi and J.-J.~L\'{e}vy, \textsl{The geometry of optimal lambda reduction}, in 9th Annual IEEE Symp. on Logic in Computer Science (LICS), 15--26, 1992.
  
\bibitem{haghverdi}
E.~Haghverdi and P.~Scott, \textsl{Geometry of {I}nteraction and the dynamics of prood reduction: a tutorial}, in New Structures for Physics, Lecture notes in Physics \textbf{813}, 357--417, 2011.
  
\bibitem{hazewinkel}
H.~Hazewinkel, \textsl{Cofree coalgebras and multivariable recursiveness}, J. Pure Appl. Algebra 183, no. 1--3, 61--103, 2003.

\bibitem{hyland}
M.~Hyland and A.~Schalk, \textsl{Glueing and orthogonality for models of linear logic}, Theoretical Computer Science, 294: 183--231, 2003.

\bibitem{JSGoTCI}
A.~Joyal and R.~Street, \textsl{The geometry of tensor calculus I}, Advances in Math. \textbf{88}, 55--112, 1991.

\bibitem{JSGoTCII}
A.~Joyal and R.~Street, \textsl{The geometry of tensor calculus II}, 
draft available at 
\href{http://maths.mq.edu.au/~street/GTCII.pdf}{http://maths.mq.edu.au/\textasciitilde street/GTCII.pdf}

\bibitem{joyal_trace}
A.~Joyal, R.~Street and D.~Verity, \textsl{Traced monoidal categories}, Math. Proc. Camb. Phil. Soc. 119, 447--468, 1996.

\bibitem{khovdia}
M.~Khovanov, \textsl{Categorifications from planar diagrammatics}, Japanese J. of Mathematics \textbf{5}, 153--181, 2010 \href{http://arxiv.org/abs/1008.5084}{[arXiv:1008.5084]}.
  
\bibitem{lafont}
Y.~Lafont, \textsl{The {L}inear {A}bstract {M}achine}, Theoretical Computer Science, 59 (1,2):157--180, 1988.

\bibitem{lambek}
J.~Lambek and P.~J.~Scott, \textsl{Introduction to higher order categorical logic}, Cambridge Studies in Advanced Mathematics, vol. 7, Cambridge University Press, Cambridge, 1986.

\bibitem{ladia}
A.~D.~Lauda, \textsl{An introduction to diagrammatic algebra and categorified quantum $\mathfrak{sl}_2$}, Bulletin of the Institute of Mathematics Academia Sinica (New Series), Vol. \textbf{7}, No. 2, 165--270, 2012 \href{http://arxiv.org/abs/1106.2128}{[arXiv:1106.2128]}.

\bibitem{mccarthy}
J.~McCarthy, \textsl{Recursive functions of symbolic expressions and their computation by machine, Part I.}, Communications of the ACM 3.4: 184--195, 1960.

\bibitem{McNameethesis}
D.~McNamee, \textsl{On the mathematical structure of topological defects in
  {L}andau-{G}inzburg models}, MSc Thesis, Trinity College Dublin, 2009.
  
\bibitem{mellies_dia}
P.-A.~Melli\`{e}s, \textsl{Functorial boxes in string diagrams}, In Z. \'{E}sik, editor, Computer Science Logic,
volume 4207 of Lecture Notes in Computer Science, pages 1--30, Springer Berlin / Heidelberg,
2006.

\bibitem{mellies}
P-A.~Melli\`{e}s, \textsl{Categorical semantics of linear logic}, in : Interactive models of computation and program behaviour, Panoramas et Synth\`{e}ses $27$, Soci\'{e}t\'{e} Math\'{e}matique de France, 2009.

\bibitem{mellies2}
P.-A.~Melli\`{e}s, N.~Tabareau, C.~Tasson, \textsl{An explicit formula for the free exponential modality of linear logic}, in: 36th International Colloquium on Automata, Languages and Programming, July 2009, Rhodes, Greece, 2009.

\bibitem{murfet}
D.~Murfet, \textsl{Computing with cut systems}, \href{http://arxiv.org/abs/1402.4541}{[arXiv:1402.4541]}.

\bibitem{murfet_coalg}
D.~Murfet, \textsl{On Sweedler's cofree cocommutative coalgebra}, \href{http://arxiv.org/abs/1406.5749}{[arXiv:1406.5749]}.

\bibitem{pagani}
M.~ Pagani and L.~Tortora de Falco.\textsl{Strong normalization property for second order linear logic}, Theoretical Computer Science 411.2 (2010): 410--444.

\bibitem{pv}
A.~Polishchuk and A.~Vaintrob, \textsl{Chern characters and {H}irzebruch-{R}iemann-{R}och formula for matrix factorizations}, Duke Mathematical Journal 161.10: 1863--1926, 2012 \href{http://arxiv.org/abs/1002.2116}{[arXiv:1002.2116]}. 

\bibitem{schreiber}
U.~Schreiber, \textsl{Quantization via {L}inear homotopy types}, \href{http://arxiv.org/abs/1402.7041}{[arXiv:1402.7041]}.

\bibitem{scott}
D.~Scott, \textsl{Data types as lattices}, SIAM Journal of computing, 5:522--587, 1976.

\bibitem{scott_talk}
\bysame, \textsl{The {L}ambda calculus, then and now}, available on \href{http://www.youtube.com/watch?v=7cPtCpyBPNI}{YouTube} with \href{http://turing100.acm.org/lambda_calculus_timeline.pdf}{lecture notes}, 2012.

\bibitem{seely}
R.~Seely, \textsl{Linear logic, star-autonomous categories and cofree coalgebras}, Applications of categories in logic and computer science, Contemporary Mathematics, 92, 1989.

\bibitem{selinger}
P.~Selinger, \textsl{Lecture notes on the {L}ambda calculus}, \href{http://arxiv.org/abs/0804.3434}{[arXiv:0804.3434]}.

\bibitem{shirahata}
M.~Shirahata, \textsl{Geometry of {I}nteraction explained}, available \href{http://www.kurims.kyoto-u.ac.jp/~hassei/algi-13/kokyuroku/19_shirahata.pdf}{online}.

\bibitem{soare}
R.~I.~Soare, \textsl{Computability and {I}ncomputability}, in CiE 2007: Computation and Logic in the Real World, LNCS 4497, Springer, 705--715.
  
\bibitem{sweedler}
M.~Sweedler, \textsl{Hopf Algebras}, W.~A.~Benjamin, New York, 1969.

\bibitem{valiron}
B.~Valiron and S.~Zdancewic, \textsl{Finite vector spaces as model of simply-typed lambda-calculi}, \href{http://arxiv.org/abs/1406.1310v1}{[arXiv:1406.1310]}.

\bibitem{quillen}
D.~Quillen, \textsl{Rational homotopy theory}, The Annals of Mathematics, Second Series, Vol. 90, No. 2, 205--295, 1969.

\bibitem{univalent}
The Univalent Foundations Program, \textsl{Homotopy {T}ype {T}heory: {U}nivalent {F}oundations of {M}athematics}, Institute for Advanced Study (Princeton), 2013.

\bibitem{Yoshino90}
Y.~Yoshino, \emph{Cohen-{M}acaulay modules over {C}ohen-{M}acaulay rings},
  London Mathematical Society Lecture Note Series, vol. 146, Cambridge
  University Press, Cambridge, 1990. 
  
\bibitem{witten}
E.~Witten, \textsl{Topological quantum field theory}, \textsl{Communications in Mathematical Physics}, 117 (3), 353--386, 1988.

\end{thebibliography}

\end{document}